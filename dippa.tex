\documentclass[a4paper]{article}

% Encoodaus, joka sopii suomenkielellä (esim. ä ja ö)
\usepackage[utf8]{inputenc}
\usepackage[T1]{fontenc}

% Suomenkielinen tavutus
\usepackage[finnish]{babel}

% Viitteet
\usepackage{natbib}

% Otsikkojen päätteetön fontti
\usepackage{sectsty}
\allsectionsfont{\sffamily\large}

% Viitteiden merkit
\bibpunct{(}{)}{;}{a}{,}{,}

\begin{document}

\title{\small T-76.5613 Software Testing and Quality Assurance, 2011 \\ Individual essay 5 \\ \huge Ohjelmiston laatu}
\date{19.12.2011}
\author{Mikko Koski \\ mikko.koski@aalto.fi \\ 66467F}
\maketitle

\large

\section{Tässä mun dippa!}

Ohjelmiston laadun dilemmaa ei tulla luultavasti koskaan ratkaisemaan. Ohjelmistoissa on nyt ja tulee jatkossakin olemaan virheitä. Riittävän hyvän laadun viitekehys on auttanut hyväksymään tämän asian ja siirtänyt katseet olennaiseen: täydellisyyteen ei kannata edes pyrkiä, riittävän hyvä on riittävän hyvä. 

Ajatusmaailmamme ohjelmistoja kohtaan on kuitenkin muututtava, jotta yleinen ohjelmistojen laatu paranee. Käyttäjät äänestävät jo nyt jaloillaan ja ostavat tuhansien ominaisuuksien kännykän sijaan kännykän, jossa on vain olennaiset ominaisuudet jotka toimivat moitteettomasti. Uskon ja toivon, että Applen ja Googlen kaltaisten yritysten menestys rohkaisee myös muita yrityksiä tekemään vähemmän, mutta paremmin.

\end{document}